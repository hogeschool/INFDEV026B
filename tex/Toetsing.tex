\section{Assessment}
	The course is tested with two exams: a series of practical assignments, a brief oral check of the practical assignments, and a theoretical exam. The final grade is determined as follows: \\

	\texttt{if theoryGrade $\geq$ 75\% \& practicumCheckOK then return practicumGrade else return insufficient}

	This means that the theoretical knowledge is a strict requirement in order to get the actual grade from the practicums, but it does not reflect the student's level of skill and as such does not further influence the grade.

	\paragraph*{Motivation for grade}
		A professional software developer is required to be able to program code which is, at the very least, \textit{correct}.

		In order to produce correct code, we expect students to show:
		\begin{inparaenum}[\itshape i\upshape)]
			\item a foundation of knowledge about how a programming language actually works in connection with a simplified concrete model of a computer;
			\item fluency when actually writing the code.
		\end{inparaenum}

		The quality of the programmer is ultimately determined by his actual code-writing skills, therefore the final grade comes only from the practicums. The quick oral check ensures that each student is able to show that his work is his own and that he has adequate understanding of its mechanisms. The theoretical exam tests that the required foundation of knowledge is also present to avoid away of programming that is exclusively based on intuition, but which is also grounded in concrete and precise knowledge about what each instruction does.


	\subsection{Theoretical examination}
		The general shape of a theoretical exam for the course is made up of a series of highly structured open questions. In each exam the content of the questions will change, but the structure of the questions will remain the same. For the structure (and an example) of the theoretical exam, see the appendix.


	\subsection{Practical examination}
		The practical examination is based on four assignments. The assignments are due by the end of the last week of the course (Week 7 - Friday - 17.00). The assignments are handed in via GitHub, by sending an email to the lecturer with:
		
		\begin{itemize}
			\item Subject: \textbf{student number - course code - practicums}
			\item Content: \textbf{a link to the GitHub repository with your sources}			
		\end{itemize}
		
		The repository:
		
		\begin{itemize}
			\item Has as name \textbf{student number - course code - practicums}
			\item Has one directory per assignment
			\item Has one readme file per assignment which sums up the results
		\end{itemize}
		
		The final grade of the practicums must be $\geq$ 75\%. The first two assignments must be done successfully.  \textbf{provided that each practicum has a grade of at least 4}. Each practicum counts for 25\% of the total grade.
