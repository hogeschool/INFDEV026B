\section{Course program}
	The course is structured into six lecture units. The lecture units are not necessarily in a one-to-one correspondence with the course weeks.

	\subsection{Unit 1 - Review}
		The course starts with a quick review on SQL and RDBMS's:

		\paragraph*{Topics}
			\begin{itemize}
				\item Entities and relationships
				\item SQL operators
			\end{itemize}
			
	\subsection{Unit 2 - Normalization}
		Theoretical concepts behind normalization of relational models. Normalization algorithms:

		\paragraph*{Topics}
			\begin{itemize}
				\item Redundancy problem.
				\item Definition of functional dependencies.
				\item Normal forms definition.
			\end{itemize}

	\subsection{Unit 3 - Normalization algorithms}
		In this unit we cover the main normalization techniques.
		\paragraph*{Topics}
		\begin{itemize}
			\item Normalization in 1NF, 2NF, 3NF, BCNF.
			\item Exercises on normalization.
		\end{itemize}		


	\subsection{Unit 4 - Concurrency}
		The fourth lecture covers handling of potentially conflicting concurrent query execution in an ACID DBMS:

		\paragraph*{Topics}
			\begin{itemize}
				\item ACID property.
				\item Serialization
				\item Locks
				\item Deadlocks and their prevention
			\end{itemize}


	\subsection{Unit 5 - NOSQL: Map-Reduce paradigm}
		The fifth lecture covers map-reduce paradigm:

		\paragraph*{Topics}
			\begin{itemize}
				\item Map function.
				\item Reduce function.
				\item Map-Reduce is SELECT-FROM-WHERE
				\item Idea behind NoSQL
				\item LINQ and map-reduce in LINQ.
			\end{itemize}


	\subsection{Unit 6 - Graph databases}
		The sixth lecture covers a specific example of no-SQL databases, specifically graph databases:

		\paragraph*{Topics}
			\begin{itemize}
				\item Directed vs undirected graphs
				\item Adjacency list vs matrix
				\item Algorithms on graphs
				\item Case study: Neo4J
			\end{itemize}
